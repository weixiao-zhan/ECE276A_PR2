\documentclass[conference]{IEEEtran}
\IEEEoverridecommandlockouts
% The preceding line is only needed to identify funding in the first footnote. If that is unneeded, please comment it out.
\usepackage{cite}
\usepackage{amsmath,amssymb,amsfonts}
\usepackage{algorithmic}
\usepackage{graphicx}
\usepackage{textcomp}
\usepackage{xcolor}
\usepackage{subcaption}

\def\BibTeX{{\rm B\kern-.05em{\sc i\kern-.025em b}\kern-.08em
    T\kern-.1667em\lower.7ex\hbox{E}\kern-.125emX}}
\begin{document}

\title{ECE276A PR2 Report}

\author{
\IEEEauthorblockN{1\textsuperscript{st} Weixiao Zhan}
\IEEEauthorblockA{
    weixiao-zhan[at]ucsd[dot]edu}
}

\maketitle


\section{Introduction}
In this project, I implemented Simultaneous Localization and Mapping (SLAM) 
on a differential-drive robot 
equipped with a wheel encoder, an Inertial Measurement Unit (IMU), 
2-D LiDAR, and an RGBD camera.

Initially, I constructed a motion model trajectory 
based on the kinematics of the differential-drive and 
utilizing data from the wheel encoder and IMU. 
Concurrently, I constructed an observation model trajectory 
using LiDAR scans through the Iterative Closest Points (ICP) method. 
Following this, I applied a Factor Graph and loop closure techniques to 
refine and optimize the trajectory. 
Finally, I leveraged the optimized trajectory to 
generate occupancy and texture maps of the environment.

\section{Problem Formulation}

In this project, there is a mixture of 2d and 3d data and algorithm. 
The robot state at time stamp t are denote in 2d and 3d slightly differently

$O_t = \left[ \right]$

\subsection{Motion Model: IMU \& Differential-drive Kinematics}


\subsection{Observation Model: LiDAR \& ICP}

\subsection{Factor Graph and Loop Closure}

\subsection{Mapping}
\subsubsection{Occupancy}

\subsubsection{Texture}


\section{Technical Approach}
[25 pts] Technical Approach: Describe your technical approach to SLAM and texture mapping.
\subsection{Time Stamp}


\section{Results}
[15 pts] Results: Present your results, and discuss them – what worked, what did not, and why.
Analyze the impact of loop closure detection and pose graph optimization on the accuracy of the robot
trajectory estimate and the resulting map quality. Make sure your results include (a) images of the
trajectory and occupancy grid map over time constructed by your SLAM algorithm and (b) textured
maps over time. If you have videos, include them in your submission zip file and refer to them in your
report.


\end{document}
